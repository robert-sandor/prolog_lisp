\documentclass[11pt]{article}
%Gummi|065|=)
\title{\textbf{List Processing in Prolog P2}}
\author{Robert Krisztian Sandor, group 927}
\date{6.11.2015}
\begin{document}

\maketitle

\textbf{10}. Given a list of integer numbers. Remove all sub-lists formed from decreasing elements. \\

\textbf{Mathematical Models}

\[
removedec(l_1, ..., l_n) = 
\left \{
	\begin{array}{ll}
		empty list & if \; n = 0 \; or \; if \; n = 2 \; and \; l_1 > l_2 \\
		l_1 & if \; n = 1 \\
		removedec(l_3, ..., l_n) & if \; l_1 > l_2 \; and \; l_2 < l_3 \\
		removedec(l_2, ..., l_n) & if \; l_1 > l_2 \; and \; l_2 > l_3 \\
		l_1 \cup removedec(l_2, ..., l_n) & otherwise
	\end{array}
\right.
\]

\textbf{Meaning of predicates. Flow models. Source Code}

\begin{verbatim}
% remove_decreasing(L : List, R : List)
% L - list of numerical atoms
% R - resulting list
% flow model (i, i), (i, o)
remove_decreasing([], []) :- !.
remove_decreasing([E], [E]) :- !.
remove_decreasing([X, Y], []) :- X > Y, !.
remove_decreasing([X, Y, Z | T], R) :-
    X > Y,
    Y < Z,
    remove_decreasing([Z | T], R), !.
remove_decreasing([X, Y | T], R) :-
    X > Y,
    remove_decreasing([Y | T], R), !.
remove_decreasing([X | T], [X | R]) :- remove_decreasing(T, R).
\end{verbatim}

\newpage

\textbf{Examples}
\begin{verbatim}
?- remove_decreasing([1,2,3,2,1,7,6,5,4,3,2,4,5,3,2], Result).
Result = [1, 2, 4].

?- remove_decreasing([5,4,3,2,1], Result).
Result = [].
\end{verbatim}

\end{document}
