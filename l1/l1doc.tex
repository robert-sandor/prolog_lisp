\documentclass[11pt]{article}
%Gummi|065|=)
\title{\textbf{Using bactraking in Prolog P4}}
\author{Robert Krisztian Sandor, group 927}
\date{20.11.2015}

\usepackage{amsmath}

\begin{document}

\maketitle

\textbf{12}. Are given N points in a plan (through their coordinates). Determine all sub-sets of collinear points. \\

\textbf{Mathematical Models}

\[
subset(l_1, ..., l_n, c) = 
\left \{
	\begin{array}{ll}
		empty list & if \; n = 0 \; or \; c = 0 \\
		l_1 \cup subset(l_2, ..., l_n, c - 1) & if \; c > 0 \\
		subset(l_2, ..., l_n, n) & if \; c > 0
	\end{array}
\right.
\]

\[
collinear(a, b, m, n, x, y) = 
\left \{
	\begin{array}{ll}
		true & if \; (n-b)*(x-m)=(y-n)*(m-a) \\
		false & otherwise
	\end{array}
\right.
\]

\begin{multline*}
getcol(x_1, x_2, y_1, y_2, p_{11}, p_{12} ..., p_{n1}, p_{n2}) = \\ 
\left \{
	\begin{array}{ll}
		(x_1, x_2, y_1, y_2) & if \; n = 0 \\
		(p_{11}, p_{12}) \cup getcol(x_1, x_2, y_1, y_2, p_{21}, p_{22} ..., p_{n1}, p_{n2}) &
		if \; collinear(x_1, x_2, y_1, y_2, p_{11}, p_{12}) \\
		getcol(x_1, x_2, y_1, y_2, p_{21}, p_{22} ..., p_{n1}, p_{n2}) & otherwise
	\end{array}
\right.
\end{multline*}

\begin{multline*}
genlist((p_{11}, p_{12}), ..., (p_{n1}, p_{n2})) = getcol((x_1, x_2), (y_1, y_2), (p_{11}, p_{12}) ..., (p_{n1}, p_{n2})) \\
where \; ((x_1, x_2), (y_1, y_2)) = subset((p_{11}, p_{12}), ..., (p_{n1}, p_{n2}))
\end{multline*}

\newpage

\textbf{Meaning of predicates. Flow models. Source Code}

\begin{verbatim}
% subset(L : list, N : Integer, R : List, Rest : List)
% L - list of atoms
% N - length of the resulting list
% R - resulting list
% Rest - L without the elements from R
% flow model (i, i, o, o)
subset([],0,[],[]).
subset([H | T], N, [H | R], Rest) :-
    N1 is N - 1,
    subset(T, N1, R, Rest).
subset([H | T], N, R, [H | Rest]) :-
    subset(T, N, R, Rest).

% collinear(X : List, Y : List, Z : List)
% X, Y, Z - list of two numerical atoms, representing a point in plane
% true if X, Y, Z are collinear, false otherwise
% flow model (i, i, i)
collinear([A, B], [M, N], [X, Y]) :-
    P is (N - B)*(X - M),
    Q is (Y - N)*(M - A),
    P is Q.

% getcol(X : List, Y : List, L : List, R : List)
% X, Y - list of two numerical atoms, representing a point in plane
% L - list of lists of two numerical atoms, representing points in plane
% R - list of points collinear to X and Y
% flow model (i, i, i, o), (i, i, i, i)
getcol(X, Y, [], [X, Y]) :- !.
getcol(X, Y, [H | T], [H | R]) :-
    collinear(X, Y, H),
    getcol(X, Y, T, R).
getcol(X, Y, [_ | T], R) :-
    getcol(X, Y, T, R).

% genlist(L : List, R : List)
% L - list of lists of two numerical atoms, representing points in plane
% R - resulting list
% flow model (i, o) 
genlist(L, R) :-
    subset(L, 2, [X, Y], Rest),
    getcol(X, Y, Rest, R).
\end{verbatim}

\textbf{Examples}
\begin{verbatim}
?- genlist([[1,2],[2,4],[3,6],[4,5],[8,10]], Result).
Result = [[3, 6], [1, 2], [2, 4]] ;
Result = [[1, 2], [2, 4]] ;
Result = [[2, 4], [1, 2], [3, 6]] ;
Result = [[1, 2], [3, 6]] ;
Result = [[1, 2], [4, 5]] ;
Result = [[1, 2], [8, 10]] ;
Result = [[1, 2], [2, 4], [3, 6]] ;
Result = [[2, 4], [3, 6]] ;
Result = [[2, 4], [4, 5]] ;
Result = [[2, 4], [8, 10]] ;
Result = [[3, 6], [4, 5]] ;
Result = [[3, 6], [8, 10]] ;
Result = [[4, 5], [8, 10]] ;
false.
\end{verbatim}

\end{document}
