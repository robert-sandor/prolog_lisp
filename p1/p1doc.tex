\documentclass[11pt]{article}
%Gummi|065|=)
\title{\textbf{Prolog Lists P1}}
\author{Robert Krisztian Sandor, group 927}
\date{8.10.2015}
\begin{document}

\maketitle

\textbf{9}. a. Transform a list in a set, in the order of the last appearance. E.g.: [1,2,3,1,2] is transformed in [3,1,2]. \\

b. Define a predicate to determine the greatest common divisor of all numbers in a list. \\

\textbf{Mathematical Models}

\[
exists(e, l_1, ..., l_n) = 
\left \{
	\begin{array}{ll}
		false & if \; n = 0 \\
		true & if \; e = l_1 \\
		exists(e, l_2, ..., l_n) & otherwise
	\end{array}
\right.
\]

\[
transform(l_1, ..., l_n) = 
\left \{
	\begin{array}{ll}
		transform(l_2, ..., l_n) & if \; exists(l_1, l_2, ..., l_n) \\
		l_1 \cup transform(l_2, ..., l_n) & otherwise
	\end{array}
\right.
\]

\textbf{Meaning of predicates. Flow models. Source Code}

\begin{verbatim}
% exists(L : List, E : Atom)
% L - list of atoms
% E - an atom
% true - if E exists in L, false otherwise
% flow model (i, i)
exists([H | _], E) :- E = H, !.
exists([_ | T], E) :- exists(T, E).

% transform(L : List, R : Set)
% L - list of atoms
% R - resulting set
% flow model (i, i), (i, o)
transform([], []).
transform([H | T], R) :-
    exists(T, H), !,
    transform(T, R).
transform([H | T], [H | R]) :- transform(T, R).

% gcd(X : Integer, Y : Integer, D : Integer)
% X, Y - numerical atoms
% D - gcd of X and Y
% flow model (i, i, i), (i, i, o)
gcd(X, 0, X) :- !.
gcd(0, X, X) :- !.
gcd(X, Y, D) :- X =< Y, !, Z is Y - X, gcd(X, Z, D).
gcd(X, Y, D) :- gcd(Y, X, D).

% gcd_list(L : List, R : Integer)
% L - list of numerical atoms
% R - gcd of all numerical atoms from L
% flow model (i, i), (i, o)
gcd_list([E], E) :- !.
gcd_list([H1, H2 | T], R) :- gcd(H1, H2, D), gcd_list([D | T], R).
\end{verbatim}

\textbf{Examples}
\begin{verbatim}
?- exists([1,2,3],1).
true.

?- exists([1,2,3],4).
false.

?- transform([1,2,3,1,2],Set).
Set = [3, 1, 2].

?- gcd(32, 8, Div).
Div = 8.

?- gcd_list([2, 4, 6, 8, 16, 20], Div).
Div = 2.

\end{verbatim}

\end{document}
