\documentclass[11pt]{article}
%Gummi|065|=)
\title{\textbf{Prolog Lists P2}}
\author{Robert Krisztian Sandor, group 927}
\date{22.10.2015}
\begin{document}

\maketitle

\textbf{2}. Determine the predecessor of a number represented as digits in a list. \\
E.g.: [1 9 3 6 0 0] to [1 9 3 5 9 9] \\

\textbf{Mathematical Models}

\[
pred(l_1, ..., l_n, c) = 
\left \{
	\begin{array}{ll}
		9 & if \; n = 1 \; and \; l_1 = 0 \\
		l_1 - 1 & if \; n = 1 \\
		pred(l_1, nc) \cup pred(l_2, ..., l_n, nc) & if \; c = 1 \\
		l_1 \cup pred(l_2, ..., l_n, nc) & otherwise
	\end{array}
\right.
\]

\[
predecessor(l_1, ..., l_n) = pred(l_1, ..., l_n, 0)
\]

\textbf{Meaning of predicates. Flow models. Source Code}

\begin{verbatim}
% predecessor_(L : List, C : Integer, R : List)
% L - list of numerical atoms (only digits)
% C - numerical atom representing a carry
% R - the resulting list
% flow model (i, i, i), (i, o, o), (i, i, o), (i, o, i)
predecessor_([E], 1, [9]) :- E is 0, !.
predecessor_([E], 0, [R]) :- R is E - 1, !.
predecessor_([H | T], C, [R | RT]) :-
    predecessor_(T, 1, RT), !,
    predecessor_([H], C, [R]).
predecessor_([H | T], 0, [H | R]) :- predecessor_(T, 0, R).

% predecessor(L : List, R : List)
% wrapper for predecessor_
% L - list of numerical atoms (only digits)
% R - the resulting list
% flow model (i, i), (i, o)
predecessor(L, R) :- predecessor_(L, 0, R).
\end{verbatim}

\textbf{Examples}
\begin{verbatim}
?- predecessor([1,9,3,6,0,0], Pred).
Pred = [1, 9, 3, 5, 9, 9].

?- predecessor([1,9,3,6,0,5], Pred).
Pred = [1, 9, 3, 6, 0, 4].

\end{verbatim}

\end{document}
